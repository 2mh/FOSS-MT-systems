% Time-stamp: <2010-11-18T00:58:22 mxp>
% -*- coding: iso-8859-1 -*-
\documentclass[11pt,twoside]{mparticle}
\usepackage[utf8]{inputenc}
\usepackage[T1]{fontenc}
\usepackage[english, ngerman]{babel}
\usepackage{textcomp}
\usepackage{mathptmx}
\usepackage[scaled]{helvet}
\usepackage[numbers]{natbib}
\usepackage{hyperref}
\usepackage{verbatim}

% Avoid widows and orphans
\widowpenalty50000
\clubpenalty50000

% Override the macros defined in the BibTeX style
\newcommand{\doi}[1]{doi: %
  \href{http://dx.doi.org/#1}{\urlstyle{rm}\nolinkurl{#1}}}
\newcommand{\eric}[1]{ERIC \href{http://eric.ed.gov/ERICWebPortal/contentdelivery/servlet/ERICServlet?accno=#1}{#1}}

% URL with access date
\newcommand{\urld}[2][???]{\url{#2} (\iflanguage{ngerman}{letzter
    Zugriff:}{accessed} #1)}

% Examples for the definition of convenience commands
\newcommand{\package}[1]{\texttt{#1}}
\newcommand{\first}[1]{\emph{#1}}
\newcommand{\q}[1]{\iflanguage{ngerman}{\flqq#1\frqq}{``#1''}}

% Title page information
\title{Open Source-MÜ-Systeme\\
  \large Referat im Seminar \q{Maschinelle \"Ubersetzung} (HS 2011)}
\author{Simon Hafner \& Hernani Marques}
\date{01. Dezember 2011}

\begin{document}
\maketitle

\begin{comment}

Systeme:
--------
URL: http://www.statmt.org/moses/
URL: http://en.wikipedia.org/wiki/Moses_%28machine_translation%29

OpenMaTrEx:
* EBMT
URL: http://openmatrex.org/

OpenLogos
* RBMT
* GPL
URL: http://de.wikipedia.org/wiki/OpenLogos (Mit Links zu Mailinglisten usw.)
URL: http://logos-os.dfki.de/
URL: http://sourceforge.net/apps/mediawiki/openlogos-mt/index.php?title=Main_Page
URL: http://sourceforge.net/projects/openlogos-mt/

Weitere: Siehe letconex.blogspot-Link unten

Technologien:
-------------
* Shallow-Transfer: Kein Wissen über den Text vorausgesetzt
-- Bei SBMT üblich
* Deep-

Generell, Vorteile FOSS-Systeme:
--------------------------------
- Nicht (nur) wirtschaftliche Kriterien sind massgebend für die Entwicklungsarbeit

Wie testen wir?
----------------
Denkbares, Testmaterial (abstrakt; vgl. Folien vom 29.9.2011 S. 9):
* Mehrdeutigkeiten
* Syntaktische Diskrepanzen
* Idiomatische Ausdrücke
* Koreferenzauflösung

Links:
------
URL: http://www.dlsi.ua.es/~mlf/docum/forcada06p2.pdf
* Combining Machine Translation Output with Open Source
URL: http://ufal.mff.cuni.cz/pbml/93/art-heafield-lavie.pdf
* Comparison of machine translation applications
URL: http://en.wikipedia.org/wiki/Comparison_of_machine_translation_applications
* Preparing Training Data (Moses)
URL: http://www.statmt.org/moses/?n=FactoredTraining.PrepareTraining
* Klingon texts
URL: http://teresh.tdonnelly.org/klintext.html
* The Klingon Language Version of the World English Bible Psalms
URL: http://ia600509.us.archive.org/3/items/TheKlingonLanguageVersionOfTheWorldEnglishBiblePsalms/kp.html#k80
* Training sessions on Open Source Machine Translation: Yielding Moses
URL: http://www.crosslang.com/training-sessions-open-source-machine-translation
* Multi-Engine Machine Translation with an Open-Source (SMT) Decoder
URL: http://www.dfki.de/lt/publication_show.php?id=2643
* Second International Workshop: On Free/Open-Source Rule-Based Machine Translation
URL: http://www.uoc.edu/freerbmt11/
** Proceedings: URL: http://openaccess.uoc.edu/webapps/o2/handle/10609/4661/browse?type=title&submit_browse=T%C3%ADtol
* Machine Translation (Wikipedia)
URL: http://en.wikipedia.org/wiki/Machine_translation
* Free/open-source machine translation software
URL: http://letconex.blogspot.com/2011/11/freeopen-source-machine-translation.html
* Open Source Machine Translation: From tools, to tricks, to projects: build a translation engine from Klingon to Finnish in an hour
URL: http://chaosradio.ccc.de/23c3_m4v_1701.html
URL: http://events.ccc.de/congress/2006/Fahrplan/events/1701.en.html
* Comparison of machine translation applications
URL: http://en.wikipedia.org/wiki/Comparison_of_machine_translation_applications
* Wikipedia Machine Translation Project
URL: http://meta.wikimedia.org/wiki/Wikipedia_Machine_Translation_Project
\end{comment}

\begin{abstract}
Wir beschäftigen uns in unserem Referat mit maschinellen Übersetzungssystemen
(MÜ-Systemen), deren Quellcode als Open Source Software (OSS) frei verfügbar ist. 
Wir möchten thematisieren, weshalb OSS-MÜ-Systeme Vorteile
gegenüber kommerziellen oder MÜ-Systemen ohne frei verfügbaren Quellcode (Closed Source Software) aufweisen.
Den (praktischen) Fokus legen wir auf zwei Systeme: Apertium und Moses. Insbesondere im
Bereich von Minderheitensprachen (z. B. Klingonisch) versprechen OSS-MÜ-Systeme dazu beizutragen diese 
im Web sichtbarer zu machen und somit das Verständnis über ihre Funktionsweise zu erhöhen und dazu
beizutragen, dass diese Sprachen (in ihrer Nutzung) an Bedeutung gewinnen.
\end{abstract}

\section{Einleitung}
\label{einleitung}
Wir legen zunächst die Problematik (Kap. \ref{problemstellung}) dar, die wir mit Closed Source-MÜ-Systemen sehen. 
In einem weiteren Schritt präsentieren wir MÜ-Systeme,
die quelloffen sind, wobei wir uns auf nur zwei Systeme beschränken (Kap. \ref{apertiumMoses}), 
von denen eines regelbasiert, das andere (primär) statistisch-basiert ist. Wir führen diese Systeme praktisch vor. Gegenstand von
Kap. \ref{herausforderungen} sind die Herausforderungen und (unsere) möglichen Lösungsansätze.
Unsere Lektüreempfehlungen finden sich am Schluss (Kap. \ref{lektuere}) dieses Handouts.

\section{Problemstellung}
\label{problemstellung}
Closed Source Software (CSS) ist Software, deren Quellcode unter Verschluss gehalten wird. Viele bekanntere MÜ-Systeme,
seien sie webbasiert, wie Google Translate\footnote{\urld[2011-11-24]{http://translate.google.com/}}, oder für den stationären persönlichen oder unternehmensweiten Einsatz gedacht, 
wie MÜ-Software von Linguatec\footnote{\urld[2011-11-24]{http://www.linguatec.de/products/}}, sind im Quellcode nicht verfügbar. 
\\
\\
Gerade bei kommerziellen Systemen werden in der Regel nur verbreitete Sprachpaare unterstützt. Das hat wirtschaftliche
Gründe (Rentabilität). Hat man zum Ziel ein Übersetzungssystem für eine Minderheitensprache zu entwickeln, wie beispielsweise 
Klingonisch, so steht man bei CSS-MÜ-Systemen üblicherweise vor verschlossenen Türen.
\\
\\
Das Problem der Transparenz stellt sich aber nicht nur hinsichtlich der Software an und für sich. Eine weitere Dimension
stellen die linguistischen Ressourcen dar, wie bei regelbasierten Ansätzen die Transferregeln, das verwendete Lexikon und die dazugehörige 
Annotation. Es ist technisch möglich diese Ressourcen einem CSS-System zu entlocken, allerdings mögen dies die
Lizenzbedingungen untersagen. Bei statistisch-basierten Systemen ist in aller Regel nicht transparent einsehbar,
welches Korpus sich hinter dem System verbirgt. Dennoch: Sind bei einem CSS-System wenigstens die Ressourcen im Sinne von
Open Source (frei) nutzbar oder ersetzbar, so ist der Spielraum bereits grösser (vgl. \cite{forcada} ab Kapitel 2.1).

\section{Herausforderungen \& Lösungsansätze}
\label{herausforderungen}
Eine Minderheitensprache zu implementieren, ist bei CSS schwierig, da
zu jeder Sprache wenigstens ein paar Regeln gehören. Im Falle von
Apertium kann man ein Kuhdorf-Dialekt von Katalanisch mit relativ
wenig Aufwand hinzufügen. Dies würde vielleicht sogar mit einer etwas
angepassten Wortliste erreicht, da die Syntax vermutlich sehr
ähnlich wäre. Bei Sprachen, die stärker abweichen, wie z. B. Klingonisch\footnote{\urld[2011-11-24]{http://www.coli.uni-saarland.de/fs-coli/procs/clauss.html}},
das stark agglutinierend ist, benötigen wir jedoch mindestens eine einfache 
Syntaxerkennung, um halbwegs brauchbare Ergebnisse zu erzielen. Diese in eine OSS zu
implementieren, sollte sich wesentlich einfacher gestalten als bei
einer CSS.
\\
\\
Ein Problem besteht für Minderheitensprachen generell: Es liegt wenig Sprachmaterial vor, weil die Sprache
in der (heutigen) Hauptressource des Webs wenig verbreitet ist. Gleichzeitig kann der OSS-Ansatz mit seiner
offenen Austauschkultur dazu beitragen, dass dieses Wissen gemehrt wird, wenn die Entwicklung öffentlich
erfolgt. Sind schliesslich MÜ-Systeme verfügbar, welche von Minderheitensprachen in weiter verbreite
Sprachen, wie Deutsch oder Englisch, übersetzen können, so kann die Sichtbarkeit der Minderheitensprachen
und damit auch die Verfügbarkeit linguistischen Sprachmaterials erhöht werden; denn: es wird Autoren möglich in der
Minderheitensprache Texte zu verfassen, im Wissen darüber, dass eine Übersetzung für grössere Leserschaften
(anderer Sprachen) möglich ist (vgl. \cite{forcada} ab Kapitel 5).

\section{Zwei OSS-MÜ-Systeme: Apertium und Moses}
\label{apertiumMoses}
Im Folgenden stellen wir zwei OSS-MÜ-Systeme vor: Apertium und Moses.

\subsection{Apertium}
\label{apertium}

\subsubsection{Abriss}
Apertium\footnote{\urld[2011-11-24]{http://www.apertium.org/}} ist ein RBMT-System, das 28 Sprachpaare 
(teilweise in bidirektionaler Übersetzungsrichtung) beherrscht. Es fällt auf, dass darunter
Minderheitensprachen zu finden sind, die (wirtschaftlich) wenig rentabel scheinen -
u. a.: Katalanisch, Esperanto, Baskisch. Finanziell unterstützt wird das Projekt von der spanischen
Zentralregierung. Die Software mit allen zugehörigen
linguistischen Ressourcen untersteht der General Public License\footnote{\urld[2011-11-24]{http://www.gnu.org/copyleft/gpl.html}} (GPL), einer Lizenz, die von der Free Software
Foundation stammt. Diese erlaubt es die Software in veränderter Form zu vertreiben - unter
der wichtigen Bedingung, dass auch Einsicht in den Quellcode möglich ist. Einige linguistischen Daten sind mit 
Creative Commons-Lizenzen\footnote{\urld[2011-11-04]{http://creativecommons.org/licenses/}} belegt, welche
zumindest eine Nutzung für nicht-kommerzielle Zwecke erlauben.

\subsubsection{Technologien}
Im Folgenden erfolgt eine Auswahl von drei Technologien, die Apertium einsetzt:
\begin{itemize}
	\item[\emph{Shallow-Transfer}] Im einfachsten Fall sucht Apertium im Quellsprachtext nach Sequenzmustern von
	Wörtern, die in einem Katalog festgehalten sind; die Übersetzung erfolgt direkt (in einem
	Schritt) anhand eines Bilexikons in die Zielsprache. Dieses Verfahren, bekannt als ``Shallow-Transfer'',
	funktioniert nur bei syntaxähnlichen Sprachen (vgl. \cite{forcadaDoc} S.73ff.)
	\item[\emph{Advanced Transfer}] Seit das Sprachpaar Englisch-Katalanisch unterstützt wird, erfolgt die
	Übersetzung in einem (fortgeschrittenen) 3-Schritte-Prozess (vgl. ebd. S.77ff.):
	\begin{enumerate}
	\item Im \emph{Chunking} werden Satzkonstituenten der Quellsprache erfasst, die Wort-für-Wort-Übersetzung
        betrieben und ferner morphosyntakische Anpassungen für die Zielsprache markiert.
	\item Es erfolgt im \emph{Interchunking} eine Neuordnung der Satzkonstituenten gemäss der Zielsprache. 
	\item Abschliessend werden mittels \emph{Postchunking} Anpassungen der Wortformen (gemäss den Markierungen
        des 2. Schritts) vorgenommen und der Text in die Zielsprache geschrieben (validiert vom Sprachgenerator).
	\end{enumerate}
	\item[\emph{XML}] Die linguistischen Daten werden im XML-Format gehalten (vgl. ebd. S.114ff.), aus Gründen
	           der Interoperabilität. So sind diese auch für andere MÜ-Systeme (theoretisch) verwendbar. Bei
		   CSS-Systemen wird im Gegenzug oftmals (bewusst) auf proprietäre, nicht-standardisierte Formate gesetzt.
\end{itemize}

\subsection{Moses}
\label{moses}

\subsubsection{Abriss}
Moses\footnote{\urld[2011-11-24]{http://www.statmt.org/moses/}} ist primär ein SBMT-System. 
Da es unter der LGPL\footnote{\urld[2011-11-24]{http://www.gnu.org/licenses/lgpl.html}} lizenziert ist, kann es als (kostenfreie) Basis für
kommerzielle Produkte verwendet werden. Es ist z. B. denkbar eigene Korpora einzubinden und
ein GUI hinzuzufügen, um in der Folge das Ganze als benutzerfreundliches Paket zu verkaufen.
Der interessante Teil ist zudem,
dass es die Möglichkeit bietet, eine Minderheitensprache zu
implementieren. Es wird bereits eine betrachtliche Anzahl von
Sprachpaaren unterstützt, die eine umfassende Anwendung ermöglichen.

\subsubsection{Technologien}
\begin{itemize}
  \item[\emph{Beam search}] Moses verwendet den ``Beam search''-Algorithmus, um im Rahmen
    seiner SBMT-Ansätzen effizient zu einem Ergebnis zu kommen
    (vgl. \cite{norvig}). Dieser Algorithmus ist eine Mischung aus den Algorithmen ``A*''
    und ``Breadth-first search'' mit reduziertem Speicherverbrauch - im Vergleich zu ``Best-first''.
  \item[\emph{Phrase-based}] Viele Wordfolgen werden direkt aus einem Korpus
    übernommen, wenn es ein Beispiel gibt; das sind Ansätze aus der EBMT (Example-Based
    Machine Translation).
  \item[\emph{Factored}] Moses unterstützt mehrdimensionale Daten für einzelne
    Tokens, was es möglich macht, manuell Regeln zu implementieren, um
    dem System bei stark flektierenden bzw. agglutinierenden Sprachen auf die
    Sprünge zu helfen. Dies ist bei z.B. Quechua / Kingonisch extrem hilfreich.
\end{itemize}

\section{Lektüreempfehlungen}
\label{lektuere}
Als wichtigste Lektüre empfehlen wir den interessanten Text ``Open-source
machine translation: an opportunity for minor languages'', welcher als Leitartikel
für unseren Beitrag gelten kann. Der Artikel ist im PDF verfügbar (vgl. \cite{forcada}).
Wir empfehlen ferner die Webseiten von Apertium und Moses zu durchstöbern, um Einblick in die
Welt von OSS-MÜ-Systemen zu erhalten:
\begin{itemize}
\item \url{http://www.apertium.org/} (Mit Web-Übersetzer)
\item \url{http://www.statmt.org/moses}
\end{itemize}
Wir wünschen viel Spass!
%%%%%%%%%%%%%%%%%%%%%%%%%%%%%%%%%%%%%%%%%%%%%%%%%%%%%%%%%%%%%%%%%%%%%%%%%%%%%%%
% Bibliography

\bibliographystyle{plain-de} % for German
% \bibliographystyle{mpplainnat} % for English
%\bibliography{seminar}
\begin{thebibliography}{...............}
\bibitem[Norvig1992]{norvig}
Norvig, Peter: \emph{Paradigms of artificial intelligence programming: case studies in common LISP}. S.195-196. San Fransisco: Morgan.
\bibitem[Forcada2010]{forcadaDoc}
Forcada Mikel L. et al.: \emph{Documentation of the Open-Source Shallow-Transfer Machine Translation Platform Apertium}.\\
URL: \urld[2011-11-24]{http://xixona.dlsi.ua.es/~fran/apertium2-documentation.pdf}

\bibitem[Forcada2006]{forcada}
Forcada, Mikel L.: \emph{Open-source machine translation: an opportunity for minor languages}. In: Strategies for developing machine translation for minority languages (5th SALTMIL workshop on Minority Languages).\\URL: \urld[2011-11-24]{http://www.dlsi.ua.es/~mlf/docum/forcada06p2.pdf}

\end{thebibliography}

\end{document}
