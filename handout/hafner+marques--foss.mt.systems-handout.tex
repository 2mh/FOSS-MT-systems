% Time-stamp: <2010-11-18T00:58:22 mxp>
% -*- coding: iso-8859-1 -*-
\documentclass[11pt,twoside]{mparticle}
\usepackage[utf8]{inputenc}
\usepackage[T1]{fontenc}
\usepackage[english, ngerman]{babel}
\usepackage{textcomp}
\usepackage{mathptmx}
\usepackage[scaled]{helvet}
\usepackage[numbers]{natbib}
\usepackage{hyperref}

% Avoid widows and orphans
\widowpenalty50000
\clubpenalty50000

% Override the macros defined in the BibTeX style
\newcommand{\doi}[1]{doi: %
  \href{http://dx.doi.org/#1}{\urlstyle{rm}\nolinkurl{#1}}}
\newcommand{\eric}[1]{ERIC \href{http://eric.ed.gov/ERICWebPortal/contentdelivery/servlet/ERICServlet?accno=#1}{#1}}

% URL with access date
\newcommand{\urld}[2][???]{\url{#2} (\iflanguage{ngerman}{letzter
    Zugriff:}{accessed} #1)}

% Examples for the definition of convenience commands
\newcommand{\package}[1]{\texttt{#1}}
\newcommand{\first}[1]{\emph{#1}}
\newcommand{\q}[1]{\iflanguage{ngerman}{\flqq#1\frqq}{``#1''}}

% Title page information
\title{Open Source-MÜ-Systeme\\
  \large Referat im Seminar \q{Maschinelle \"Ubersetzung} (HS 2011)}
\author{Simon Hafner \& Hernani Marques}
\date{01. Dezember 2011}

\begin{document}
\maketitle

\begin{verbatim}

Systeme:
--------
Anusaaraka:
* RBMT
* GPL
URL: http://anusaaraka.iiit.ac.in/

Apertium:
* GPL
* RBMT
** Siehe Folien vom 29.9.2011 (S. 8: Chart darüber, wie Apertium (grob) funktioniert)
** Genauer ausführen
URL: http://www.apertium.org/

Matxin
* GPL
* RBMT
* Nur erwähnen
URL: http://matxin.sf.net/

Moses:
* LGPL
** Kann als Basis für kommerzielle Produkte genutzt werden
** Geringe Total Cost of Ownership (CrossLang-URL)
** Geringe Time-To-Market (CrossLang-URL)
** Minderheitssprachen
Möglichkeit zur HAMT um Text in der Sprache zu generieren.
** Klingonisch
Bräuchte garantiert Morphologie, weil klingonisch sehr agglutinierend
ist, sparse data Problem.
** Aufbau
- beam
- phrase-based (example-based)
- factored (see klingon)
- confusion networks (WFST)
* SBMT
URL: http://www.statmt.org/moses/
URL: http://en.wikipedia.org/wiki/Moses_%28machine_translation%29

OpenMaTrEx:
* EBMT
URL: http://openmatrex.org/

OpenLogos
* RBMT
* GPL
URL: http://de.wikipedia.org/wiki/OpenLogos (Mit Links zu Mailinglisten usw.)
URL: http://logos-os.dfki.de/
URL: http://sourceforge.net/apps/mediawiki/openlogos-mt/index.php?title=Main_Page
URL: http://sourceforge.net/projects/openlogos-mt/

Weitere: Siehe letconex.blogspot-Link unten

Technologien:
-------------
* Shallow-Transfer: Kein Wissen über den Text vorausgesetzt
-- Bei SBMT üblich
* Deep-

Generell, Vorteile FOSS-Systeme:
--------------------------------
- Nicht (nur) wirtschaftliche Kriterien sind massgebend für die Entwicklungsarbeit

Wie testen wir?
----------------
Denkbares, Testmaterial (abstrakt; vgl. Folien vom 29.9.2011 S. 9):
* Mehrdeutigkeiten
* Syntaktische Diskrepanzen
* Idiomatische Ausdrücke
* Koreferenzauflösung

Links:
------
* Open-source machine translation: an opportunity for minor languages
URL: http://www.dlsi.ua.es/~mlf/docum/forcada06p2.pdf
* Combining Machine Translation Output with Open Source
URL: http://ufal.mff.cuni.cz/pbml/93/art-heafield-lavie.pdf
* Comparison of machine translation applications
URL: http://en.wikipedia.org/wiki/Comparison_of_machine_translation_applications
* Preparing Training Data (Moses)
URL: http://www.statmt.org/moses/?n=FactoredTraining.PrepareTraining
* Klingon texts
URL: http://teresh.tdonnelly.org/klintext.html
* The Klingon Language Version of the World English Bible Psalms
URL: http://ia600509.us.archive.org/3/items/TheKlingonLanguageVersionOfTheWorldEnglishBiblePsalms/kp.html#k80
* Training sessions on Open Source Machine Translation: Yielding Moses
URL: http://www.crosslang.com/training-sessions-open-source-machine-translation
* Multi-Engine Machine Translation with an Open-Source (SMT) Decoder
URL: http://www.dfki.de/lt/publication_show.php?id=2643
* Second International Workshop: On Free/Open-Source Rule-Based Machine Translation
URL: http://www.uoc.edu/freerbmt11/
** Proceedings: URL: http://openaccess.uoc.edu/webapps/o2/handle/10609/4661/browse?type=title&submit_browse=T%C3%ADtol
* Machine Translation (Wikipedia)
URL: http://en.wikipedia.org/wiki/Machine_translation
* Free/open-source machine translation software
URL: http://letconex.blogspot.com/2011/11/freeopen-source-machine-translation.html
* Open Source Machine Translation: From tools, to tricks, to projects: build a translation engine from Klingon to Finnish in an hour
URL: http://chaosradio.ccc.de/23c3_m4v_1701.html
URL: http://events.ccc.de/congress/2006/Fahrplan/events/1701.en.html
* Comparison of machine translation applications
URL: http://en.wikipedia.org/wiki/Comparison_of_machine_translation_applications
* Wikipedia Machine Translation Project
URL: http://meta.wikimedia.org/wiki/Wikipedia_Machine_Translation_Project
\end{verbatim}

\section{Abstract}
\label{abstract}
In unserem Referat möchten wir einen Überblick über MÜ-Systeme

\section{Einleitung}
\label{grundlagen}
\begin{verbatim}
- sparse-data-Problem
\end{verbatim}

\section{Überblick}
\label{ueberblick}
\begin{verbatim}
- sparse-data-Problem
\end{verbatim}


\section{Problemstellung}
\label{problemstellung}
\begin{verbatim}
- sparse-data-Problem
\end{verbatim}


\section{Begriffe \& Grundlagen}
\label{grundlagen}
\begin{verbatim}
- sparse-data-Problem
\end{verbatim}

\section{Herausforderungen \& -lösungsansätze}
\label{herausforderungen}
\begin{verbatim}
- sparse-data-Problem
\end{verbatim}

\section{Zusammenfassung}
\label{zusammenfassung}

\section{Lektüreempfehlungen}
\label{lektuere}

%%%%%%%%%%%%%%%%%%%%%%%%%%%%%%%%%%%%%%%%%%%%%%%%%%%%%%%%%%%%%%%%%%%%%%%%%%%%%%%
% Bibliography

\bibliographystyle{plain-de} % for German
% \bibliographystyle{mpplainnat} % for English
\bibliography{seminar}
\end{document}
