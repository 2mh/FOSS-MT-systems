% Time-stamp: <2010-11-18T00:58:22 mxp>
% -*- coding: iso-8859-1 -*-
\documentclass[11pt,twoside]{mparticle}
\usepackage[utf8]{inputenc}
\usepackage[T1]{fontenc}
\usepackage[english, ngerman]{babel}
\usepackage{textcomp}
\usepackage{mathptmx}
\usepackage[scaled]{helvet}
\usepackage[numbers]{natbib}
\usepackage{hyperref}
\usepackage{verbatim}

% Avoid widows and orphans
\widowpenalty50000
\clubpenalty50000

% Override the macros defined in the BibTeX style
\newcommand{\doi}[1]{doi: %
  \href{http://dx.doi.org/#1}{\urlstyle{rm}\nolinkurl{#1}}}
\newcommand{\eric}[1]{ERIC \href{http://eric.ed.gov/ERICWebPortal/contentdelivery/servlet/ERICServlet?accno=#1}{#1}}

% URL with access date
\newcommand{\urld}[2][???]{\url{#2} (\iflanguage{ngerman}{letzter
    Zugriff:}{accessed} #1)}

% Examples for the definition of convenience commands
\newcommand{\package}[1]{\texttt{#1}}
\newcommand{\first}[1]{\emph{#1}}
\newcommand{\q}[1]{\iflanguage{ngerman}{\flqq#1\frqq}{``#1''}}

% Title page information
\title{Open Source-MÜ-Systeme\\
  \large Referat im Seminar \q{Maschinelle \"Ubersetzung} (HS 2011)}
\author{Simon Hafner \& Hernani Marques}
\date{01. Dezember 2011}

\begin{document}
\maketitle

\begin{comment}

Systeme:
--------
Anusaaraka:
* RBMT
* GPL
URL: http://anusaaraka.iiit.ac.in/

Apertium:
* GPL
* RBMT
** Siehe Folien vom 29.9.2011 (S. 8: Chart darüber, wie Apertium (grob) funktioniert)
** Genauer ausführen
URL: http://www.apertium.org/

Matxin
* GPL
* RBMT
* Nur erwähnen
URL: http://matxin.sf.net/

Moses:
* LGPL
URL: http://www.statmt.org/moses/
URL: http://en.wikipedia.org/wiki/Moses_%28machine_translation%29

OpenMaTrEx:
* EBMT
URL: http://openmatrex.org/

OpenLogos
* RBMT
* GPL
URL: http://de.wikipedia.org/wiki/OpenLogos (Mit Links zu Mailinglisten usw.)
URL: http://logos-os.dfki.de/
URL: http://sourceforge.net/apps/mediawiki/openlogos-mt/index.php?title=Main_Page
URL: http://sourceforge.net/projects/openlogos-mt/

Weitere: Siehe letconex.blogspot-Link unten

Technologien:
-------------
* Shallow-Transfer: Kein Wissen über den Text vorausgesetzt
-- Bei SBMT üblich
* Deep-

Generell, Vorteile FOSS-Systeme:
--------------------------------
- Nicht (nur) wirtschaftliche Kriterien sind massgebend für die Entwicklungsarbeit

Wie testen wir?
----------------
Denkbares, Testmaterial (abstrakt; vgl. Folien vom 29.9.2011 S. 9):
* Mehrdeutigkeiten
* Syntaktische Diskrepanzen
* Idiomatische Ausdrücke
* Koreferenzauflösung

Links:
------
* Open-source machine translation: an opportunity for minor languages
URL: http://www.dlsi.ua.es/~mlf/docum/forcada06p2.pdf
* Combining Machine Translation Output with Open Source
URL: http://ufal.mff.cuni.cz/pbml/93/art-heafield-lavie.pdf
* Comparison of machine translation applications
URL: http://en.wikipedia.org/wiki/Comparison_of_machine_translation_applications
* Preparing Training Data (Moses)
URL: http://www.statmt.org/moses/?n=FactoredTraining.PrepareTraining
* Klingon texts
URL: http://teresh.tdonnelly.org/klintext.html
* The Klingon Language Version of the World English Bible Psalms
URL: http://ia600509.us.archive.org/3/items/TheKlingonLanguageVersionOfTheWorldEnglishBiblePsalms/kp.html#k80
* Training sessions on Open Source Machine Translation: Yielding Moses
URL: http://www.crosslang.com/training-sessions-open-source-machine-translation
* Multi-Engine Machine Translation with an Open-Source (SMT) Decoder
URL: http://www.dfki.de/lt/publication_show.php?id=2643
* Second International Workshop: On Free/Open-Source Rule-Based Machine Translation
URL: http://www.uoc.edu/freerbmt11/
** Proceedings: URL: http://openaccess.uoc.edu/webapps/o2/handle/10609/4661/browse?type=title&submit_browse=T%C3%ADtol
* Machine Translation (Wikipedia)
URL: http://en.wikipedia.org/wiki/Machine_translation
* Free/open-source machine translation software
URL: http://letconex.blogspot.com/2011/11/freeopen-source-machine-translation.html
* Open Source Machine Translation: From tools, to tricks, to projects: build a translation engine from Klingon to Finnish in an hour
URL: http://chaosradio.ccc.de/23c3_m4v_1701.html
URL: http://events.ccc.de/congress/2006/Fahrplan/events/1701.en.html
* Comparison of machine translation applications
URL: http://en.wikipedia.org/wiki/Comparison_of_machine_translation_applications
* Wikipedia Machine Translation Project
URL: http://meta.wikimedia.org/wiki/Wikipedia_Machine_Translation_Project
\end{comment}

\begin{abstract}
Wir beschäftigen uns in unserem Referat mit maschinellen Übersetzungssystemen
(MÜ-Systemen), deren Quellcode als Open Source Software (OSS) frei verfügbar ist. 
Nach einer Einführung der verschiedenen (grösseren) Systemen, die in diesem 
Bereich existieren, möchten wir thematisieren, weshalb OSS-MÜ-Systeme Vorteile
gegenüber kommerziellen oder MÜ-Systemen ohne frei verfügbaren Quellcode (Closed Source Software) aufweisen.
Den (praktischen) Fokus legen wir auf zwei Systeme: Apertium und Moses. Insbesondere im
Bereich von Minderheitensprachen (z. B. Klingonisch) versprechen OSS-MÜ-Systeme dazu beizutragen diese 
im Web sichtbarer zu machen und somit das Verständnis über ihre Funktionsweise zu erhöhen und dazu
beizutragen, dass diese Sprachen (in ihrer Nutzung) an Bedeutung gewinnen.
\end{abstract}

\section{Einleitung}
\label{einleitung}
Wir legen zunächst (\ref{problemstellung}) die Problematik dar, die wir mit Closed Source-MÜ-Systemen sehen. 
In einem weiteren Schritt bieten wir einen Überblick (Kap. \ref{ueberblick}) über die (grösseren) MÜ-Systeme,
die Open Source-Software sind, um uns im Folgekapitel \ref{apertiumMoses} auf nur zwei Systeme zu beschränken, 
von denen eines regelbasiert, das andere statistisch-basiert ist. Wir führen diese Systeme praktisch vor. Gegenstand von
Kapitel \ref{herausforderungen} sind die Herausforderungen und (unsere) möglichen Lösungsansätze.
Unsere Lektüreempfehlungen finden sich am Schluss (Kap. \ref{lektuere}) dieses Handouts.

\section{Problemstellung}
\label{problemstellung}
Closed Source Software (CSS) ist Software, deren Quellcode unter Verschluss gehalten wird. Viele bekanntere MÜ-Systeme,
seien sie webbasiert, wie Google Translate, oder für den stationären persönlichen oder unternehmsweiten Einsatz, 
wie MÜ-Software von Linguatec, sind im Quellcode nicht verfügbar. 
\\
\\
Gerade bei kommerziellen Systemen werden in der Regel nur verbreite Sprachpaare unterstützt. Das hat wirtschaftliche
Gründe (Rentabilität). Hat man zum Ziel ein Übersetzungssystem für eine Minderheitensprache zu entwickeln, wie beispielsweise 
Klingonisch, so steht man bei CSS-MÜ-Systemen üblicherweise vor verschlossenen Türen.
\\
\\
Das Problem der Transparenz stellt sich aber nicht nur hinsichtlich der Software an und für sich. Eine weitere Dimension
stellen die linguistischen Ressourcen dar, wie bei regelbasierten Ansätzen das verwendete Lexikon und die dazugehörige 
Annotation. Es ist technisch möglich diese Ressourcen einem CSS-System zu entlocken, allerdings mögen dies die
Lizenzbedingungen untersagen. Bei statistisch-basierten Systemen ist in aller Regel nicht transparenz einsehbar,
welchen Korpus sich hinter dem System verbirgt. Sind bei einem CSS-System wenigstens die Ressourcen im Geiste von
Open Source verfügbar und (frei) nutzbar, so ist uns mehr Freiheit gegeben.

\section{Herausforderungen \& Lösungsansätze}
\label{herausforderungen}
Eine Minderheitensprache zu implementiern, ist bei CSS unmöglich, da
zu jeder Sprache wenigstens ein paar Regeln gehören. Im Falle von
Apertium kann man ein Kuhdorf-Dialekt von Katalanisch mit relativ
wenig Aufwand hinzufügen. Dies würde vielleicht sogar mit einer etwas
angepassten Wortliste erreichen, da die Grammatik vermutlich sehr
ähnlich ist. Bei Sprachen, die stärker abweichen, wie z.B. Klingonisch
benötigen wir jedoch mindestens eine einfache Syntaxerkennung um
halbwegs brauchbare Ergebnisse zu erzielen. Diese in eine OSS zu
implementieren sollte sich wesentlich einfacher gestalten als bei
einer CSS.

\section{Überblick: OSS-MÜ-Systeme}
\label{ueberblick}

\section{Zwei OSS-MÜ-Systeme: Apertium und Moses}
\label{apertiumMoses}


\subsection{Apertium}
\label{apertium}
Apertium ist ein RBMT-System, das 28 Sprachpaare 
(teilweise in bidirektionaler Übersetzungsrichtung) beherrscht. Es fällt auf, dass darunter
Minderheitensprachen zu finden sind, die (wirtschaftlich) wenig rentabel scheinen -
u. a.: Katalanisch, Esperanto, Baskisch. Die Software mit allen zugehörigen
linguistischen Ressourcen unterstehen der General Public License (GPL), einer Lizenz, die von der Free Software
Foundation stammt. Dies erlaubt es die Software in veränderter Form zu betreiben unter
der wichtigen Bedingung, dass auch Einsicht in den Quellcode möglich ist.

\subsubsection{Technologien}

\subsection{Moses}
\label{moses}
\subsubsection{Propaganda}
Da Moses unter der LGPL lizenziert ist, kann es als Basis für
kommerzielle Produkte verwendet werden. So kann man eigene Korpora und
ein GUI hinzufügen und das ganze als Paket verkaufen. Zudem entstehen
keine Kosten durch Lizenzgebühren. Der interessante Teil ist zudem,
dass es die Möglichkeit bietet, eine Minderheitssprache zu
implementieren. Es hat zudem eine betrachtliche Anzahl von
Sprachpaaren, die eine vielseitige Anwendung ermöglichen.

\subsubsection{Technologien}
\begin{itemize}
  \item[beam] Moses verwendet den beam Algorithmus, um effizient zu
    einem Ergebnis zu kommen. (SBMT)
  \item[phrase-based] Viele Sätze werden direkt aus einem Korpus
    übernommen, wenn es ein Beispiel gibt (EBMT)
  \item[factored] Zudem verwendet Moses noch Regeln, um flektierende
    und agglutinierende Sprachen wie Kingonisch auseinanderzuziehen,
    um die einzelnen Teile übersetzen zu können. (RBMT)
  \item[confusion networks] Als Input können auch WFST (weighted
    finite state transducer) verwendet werden, aus z.B. einem
    Speech2Text-System.
\end{itemize}

\section{Lektüreempfehlungen}
\label{lektuere}

%%%%%%%%%%%%%%%%%%%%%%%%%%%%%%%%%%%%%%%%%%%%%%%%%%%%%%%%%%%%%%%%%%%%%%%%%%%%%%%
% Bibliography

\bibliographystyle{plain-de} % for German
% \bibliographystyle{mpplainnat} % for English
\bibliography{seminar}
\end{document}
