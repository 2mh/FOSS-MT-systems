% Time-stamp: <2010-11-18T00:58:22 mxp>
% -*- coding: iso-8859-1 -*-
\documentclass[11pt,twoside]{mparticle}
\usepackage[utf8]{inputenc}
\usepackage[T1]{fontenc}
\usepackage[english, ngerman]{babel}
\usepackage{textcomp}
\usepackage{mathptmx}
\usepackage[scaled]{helvet}
\usepackage[numbers]{natbib}
\usepackage{hyperref}

% Avoid widows and orphans
\widowpenalty50000
\clubpenalty50000

% Override the macros defined in the BibTeX style
\newcommand{\doi}[1]{doi: %
  \href{http://dx.doi.org/#1}{\urlstyle{rm}\nolinkurl{#1}}}
\newcommand{\eric}[1]{ERIC \href{http://eric.ed.gov/ERICWebPortal/contentdelivery/servlet/ERICServlet?accno=#1}{#1}}

% URL with access date
\newcommand{\urld}[2][???]{\url{#2} (\iflanguage{ngerman}{letzter
    Zugriff:}{accessed} #1)}

% Examples for the definition of convenience commands
\newcommand{\package}[1]{\texttt{#1}}
\newcommand{\first}[1]{\emph{#1}}
\newcommand{\q}[1]{\iflanguage{ngerman}{\flqq#1\frqq}{``#1''}}

% Title page information
\title{Titel des Referats\\
  \large Referat im Seminar \q{Maschinelle \"Ubersetzung} (HS 2011)}
\author{Donald Duck}
\date{22. September 2011}

\begin{document}
\maketitle

\section{Zu den handouts}
\label{sec:zu-handouts}

Jeder sollte bis spätestens 7 tage vor seinem referatstermin einen
vorschlag für die gemeinsame lektüre und das handout einreichen.

Zusammen mit allen inhaltlichen daten ist ein ausgefülltes und
unterschriebenes exemplar der erklärung zu den grundregeln der
akademischen ehrlichkeit (im ordner \q{Materialien -- Vorlagen/Admin})
abzugeben.

Das handout (auch bekannt als exposé, thesenpapier oder disposition)
soll

\begin{itemize}
\item den titel des referat enthalten,
\item im stil eines abstracts aus allgemeiner sicht mit ca. 60 bis 100
  wörtern den gehalt des referats beschreiben und
\item den aufbau des referats in abschnitten von etwa 2 bis 7 minuten
  kurz und bündig darlegen.
\end{itemize}

Der grundsätzlich aufbau eines referats (und im prinzip jeder
wissenschaftlichen arbeit) ist:

\begin{enumerate}
\item Problem oder fragestellung
\item Lösungsansatz
\item Evaluation
\end{enumerate}

Außerdem geben Sie bitte die wichtigsten literaturreferenzen an (siehe
abschnitt~\ref{sec:referenzen}) und den vorschlag für die gemeinsame
lektüre.

\section{Die klasse \package{mparticle}}
\label{sec:mparticle}

Die klasse \package{mparticle}\footnote{Erhältlich von
  \urld[2011-09-20]{http://dynalabs.de/mxp/latex/}} wurde von Michael
Piotrowski entwickelt und kann grundsätzlich als direkter ersatz für
die \LaTeX-standardklasse \package{article} verwendet werden. Einige
unterschiede zwischen \package{mparticle} und \package{article} sind
jedoch zu erwähnen.

\package{mparticle} verwendet automatisch A4 als seitenformat, es muss
also nicht extra angegeben werden.

Der vielleicht größte unterschied ist, dass bei \package{mparticle}
bild- und tabellenunterschriften innerhalb einer \texttt{figure}- bzw.
\texttt{table}-umgebung \emph{vor} dem eigentlichen bild\footnote{Für
  die einbindung von grafiken muss das paket \package{graphicx} mit
  geladen werden.} bzw. der eigentlichen tabelle definiert werden
sollten, z.\,b.:

\begin{verbatim}
\begin{table}[!ht]
  \caption{Erläuterung zur tabelle}
  \begin{tabular}{|l|l|}
    .
    .
    .
  \end{tabular}
  \label{tab:terms}
\end{table}
\end{verbatim}

Vergleiche auch die beispiele im quelltext dieses dokuments. Die
angabe der bild- bzw. tabellenunterschrift nach dem bild bzw. der
tabelle resultiert in einem unerwünschten und unschönen
erscheinungsbild.

Wenn Sie \package{listings} oder \package{longtable} verwenden, müssen
Sie noch zusätzliche definitionen laden, damit die listing-
bzw. tabellenunterschriften an der richtigen position erscheinen.

\section{Referenzen}
\label{sec:referenzen}

Verwenden Sie für Ihre arbeit \package{natbib} für zitate.
\package{natbib} stellt verschiedene zitierbefehle bereit, die Sie je
nach kontext verwenden sollten:

\begin{itemize}
\item Verwenden Sie \verb|\citep| für parenthetische referenzen,
  z.\,b.:

  \begin{quote}
    Bereits vor den 1980er jahren gab es forschung zu
    autorenwerkzeugen, die NLP-techniken verwendete. \citep{dale1997}
  \end{quote}
\item Verwenden Sie \verb|\citet| für referenzen im text, z.\,b.:

  \begin{quote}
    \citet{stallman1981} beschreibt mit dem Emacs einen editor, bei
    dem die benutzer teile auswechseln können.
  \end{quote}
\item Verwenden Sie \verb|citeauthor|, um nur den oder die
  autorennamen zu erhalten, z.\,b.:

  \begin{quote}
    \citeauthor{knutsson2005} stellte \citeyear{knutsson2005} fest:
    \q{Writing and written language play today an increasingly
      important part in many people's lives.} \citep[S.~3]{knutsson2005}
  \end{quote}

  Mit \verb|\citeyear| kann man auf das jahr einer veröffentlichung
  zugreifen.
\end{itemize}

\package{natbib} stellt noch weitere zitierkommandos zur verfügung,
hierfür sei auf die dokumentation verwiesen (in Ihrer lokalen
\TeX-installation oder auch online\footnote{\urld[2011-09-20]{http://www.ctan.org/tex-archive/macros/latex/contrib/natbib/natbib.pdf}}).

Bei elektronischen oder elektronisch verfügbaren publikationen
verwenden Sie in Ihrer \textsc{Bib}\TeX-datei entweder das feld
\verb|doi| (für veröffentlichungen, die einen DOI haben) oder die
felder \verb|url| und \verb|urldate| zur angabe eines URL und des
letzten zugriffs.

Für URLs im text (wenn ein bibliografieeintrag nicht sinnvoll ist) ist
oben in dieser beispieldatei das makro \verb|\urld| für datierte URLs
definiert; die eingabe

\begin{verbatim}
\urld[2011-09-20]{http://adobe.com/}
\end{verbatim}

erscheint dann als

\begin{quote}
  \urld[2011-09-20]{http://adobe.com/}
\end{quote}

%%%%%%%%%%%%%%%%%%%%%%%%%%%%%%%%%%%%%%%%%%%%%%%%%%%%%%%%%%%%%%%%%%%%%%%%%%%%%%%
% Bibliography

\bibliographystyle{plain-de} % for German
% \bibliographystyle{mpplainnat} % for English
\bibliography{seminar}

\end{document}

%%% Local variables:
%%% TeX-PDF-mode: t
%%% End:
