% beamer
\documentclass{clsemhs11-beamer}
\usepackage[ngerman]{babel}
\usepackage[utf8]{inputenc}
% Führt zu Probs (noch)
%\usepackage[T1]{fontenc}
\usepackage{helvet}

\newcommand{\first}[1]{\emph{#1}}
\newcommand{\q}[1]{\iflanguage{ngerman}{\flqq#1\frqq}{``#1''}}

\title{Maschinelle \"Ubersetzung}
\subtitle{Seminar \q{Maschinelle \"Ubersetzung} (HS~2011)}
\author{Minnie Mausie}
\date{\today}

\begin{document}
  \maketitle

\begin{frame}
  \frametitle{Der Titel der ersten Folie}

  Die folgenden Punkte werden nach und nach aufgedeckt:
  
  \begin{itemize}[<+->]
  \item Foo
  \item Bar
  \item Baz
  \end{itemize}

  Diese dagegen erscheinen aufs Mal:

  \begin{itemize}
  \item Foo
  \item Bar
  \item Baz
  \end{itemize}
\end{frame}

\begin{frame}
  \frametitle{Der Titel der zweiten Folie}

  Beispiel einer Grafik:

  \includegraphics[width=\textwidth]{graphics/dilbert-unix}

  Breite und Höhe muss man je nach Format anpassen.
\end{frame}

\begin{frame}
  \frametitle{Beamer}

  Beamer hat eine irrsinnige Zahl von Optionen, siehe Handbuch:

  \url{http://mirrors.ctan.org/macros/latex/contrib/beamer/doc/beameruserguide.pdf}
\end{frame}
\end{document}

%%% Local variables:
%%% TeX-PDF-mode: t
%%% End:
