% beamer
\documentclass{beamer}
\usepackage[ngerman]{babel}
\usepackage[utf8]{inputenc}
\usepackage[T1]{fontenc}
\usepackage{helvet}
\usepackage{beamerthemesplit}
\usepackage[numbers]{natbib}

\newcommand{\first}[1]{\emph{#1}}
\newcommand{\q}[1]{\iflanguage{ngerman}{\flqq#1\frqq}{``#1''}}
\def\newblock{\hskip .11em plus.33em minus.07em}

\renewcommand{\footnotesize}{\tiny}

\title{Open Source-MÜ-Systeme}
\subtitle{Seminar \q{Maschinelle \"Ubersetzung} (HS~2011)}
\author{Simon Hafner \& Hernani Marques}
\date{\today}

\begin{document}
  \maketitle


\begin{frame}
\frametitle{Referatsübersicht}
\tableofcontents
\end{frame}

%%%%%%%%%%%%%%%%%%%%%%%%%%%%%%%%%%%%%%%%%%%%%%%%%%%%%%%%%%%%%%%%%%%%%%%%%%%%%
% SECTION
%%%%%%%%%%%%%%%%%%%%%%%%%%%%%%%%%%%%%%%%%%%%%%%%%%%%%%%%%%%%%%%%%%%%%%%%%%%%%
\section{Problemstellung}
% SUBSECTION
\subsection{Warum Closed Source vs. Open Source Software?}
% OSS (generell)
\begin{frame}
\frametitle{Open Source Software (generell)}
\begin{figure}
  \includegraphics[width=0.30\textwidth]{graphics/ossmatrix}
  \caption{Open Source Software (Logos)\footnote{URL: \url{http://cdn.lostintechnology.com/wp-content/uploads/2010/12/opensource.jpg} (30.11.2011)}}
  \end{figure}
\begin{itemize}
\item[\emph{Lizenzen}] GPL, LGPL, CC, MIT/BSD/Apache, Shared Source u. a. (libertär/viral, liberal, zweckbeschränkt)
\item[\emph{Software}] Betriebssysteme, Browser, Spiele u. a.; auch: MÜ-Systeme :)
\end{itemize}
\end{frame}
\begin{frame}
% CSS (generell)
\frametitle{Closed Source Software (generell)}
\begin{figure}
  \includegraphics[width=0.30\textwidth]{graphics/cssblackbox}
  \caption{Closed Source Software (als Blackbox verstanden)\footnote{URL: \url{http://juid.de/images/Kaleidoskop/The_black_box.jpg} (30.11.2011)}}
  \end{figure}
\begin{itemize}
\item[\emph{Lizenzen}] EULA, Mehrplatzlizenzen, Shareware, Freeware u. a.
\item[\emph{Software}] Betriebssysteme, Browser, Spiele u. a.; auch: MÜ-Systeme :)
\end{itemize}
\end{frame}
% OSS vs. CSS

% SUBSECTION
\subsection{Minderheitensprachen}
% Bilder: Esperanto, Klingonisch, Baskisch, Quechua
%%%%%%%%%%%%%%%%%%%%%%%%%%%%%%%%%%%%%%%%%%%%%%%%%%%%%%%%%%%%%%%%%%%%%%%%%%%%%
% SECTION
%%%%%%%%%%%%%%%%%%%%%%%%%%%%%%%%%%%%%%%%%%%%%%%%%%%%%%%%%%%%%%%%%%%%%%%%%%%%%
\section{Herausforderungen \& Lösungsansätze}
\subsection{Sparse data-Probleme}
%%%%%%%%%%%%%%%%%%%%%%%%%%%%%%%%%%%%%%%%%%%%%%%%%%%%%%%%%%%%%%%%%%%%%%%%%%%%%
% SECTION
%%%%%%%%%%%%%%%%%%%%%%%%%%%%%%%%%%%%%%%%%%%%%%%%%%%%%%%%%%%%%%%%%%%%%%%%%%%%%
\section{Zwei OSS-MÜ-Systeme: Apertium und Moses}
\subsection{Apertium}
\subsection{Moses}
%%%%%%%%%%%%%%%%%%%%%%%%%%%%%%%%%%%%%%%%%%%%%%%%%%%%%%%%%%%%%%%%%%%%%%%%%%%%%
% SECTION
%%%%%%%%%%%%%%%%%%%%%%%%%%%%%%%%%%%%%%%%%%%%%%%%%%%%%%%%%%%%%%%%%%%%%%%%%%%%%
\begin{frame}
  \frametitle{Lektüreempfehlungen}

\begin{thebibliography}{...............}
\small
\bibitem[Norvig1992]{norvig}
Norvig, Peter: \emph{Paradigms of artificial intelligence programming: case studies in common LISP}. S.195-196. San Fransisco: Morgan.
\bibitem[Forcada2010]{forcadaDoc}
Forcada Mikel L. et al.: \emph{Documentation of the Open-Source Shallow-Transfer Machine Translation Platform Apertium}.\\
URL: \url{http://xixona.dlsi.ua.es/~fran/apertium2-documentation.pdf} (30.11.2011)

\bibitem[Forcada2006]{forcada}
Forcada, Mikel L.: \emph{Open-source machine translation: an opportunity for minor languages}. In: Strategies for developing machine translation for minority languages (5th SALTMIL workshop on Minority Languages).\\URL: \url{http://www.dlsi.ua.es/~mlf/docum/forcada06p2.pdf} (30.11.2011)
\end{thebibliography}

\end{frame}
%%%%%%%%%%%%%%%%%%%%%%%%%%%%%%%%%%%%%%%%%%%%%%%%%%%%%%%%%%%%%%%%%%%%%%%%%%%%%
% SECTION
%%%%%%%%%%%%%%%%%%%%%%%%%%%%%%%%%%%%%%%%%%%%%%%%%%%%%%%%%%%%%%%%%%%%%%%%%%%%%
\section{Fragen}
\begin{frame}
  \frametitle{Dalegh'a'\footnote{\emph{engl.} ``Do you see it?''. URL: \url{http://en.wikibooks.org/wiki/Klingon/Grammar/Questions} (30.11.2011)}}
  \begin{figure}
  \includegraphics[width=0.40\textwidth]{graphics/xkcd--cl}
  \caption{XKCD - Nr. 114\footnote{URL: \url{http://xkcd.com/114/} (30.11.2011)}}
  \end{figure}
\end{frame}
\end{document}
